\label{setup}
\subsection{Experimental Setup}
We conduct the experimental evaluations on the proposed node assignment methods
(\emph{sequential} and \emph{distributed}) by using three different real-world
workloads; workload from Hornet Cluster at the University of Connecticut~\cite{hornet},
parallel workload from IBM-SP2 Cluster at Cornell Theory Center(CTC)~\cite{ctcsp2} and NFS
workload of a research workload from a university computer science department~\cite{lair62}.

\subsubsection{Workloads}
\paragraph{Hornet Cluster}
The Hornet Cluster is a high end cluster consisting of 64 nodes where each node has 12 Intel Xeon X5650
Westmere cores, 48 GB of RAM and 500 GB of storage. Users submit jobs on this cluster and
each job is recorded in the log files. For each job, the log files provide the identifier of the
user who submitted the job, in addition to the submission and completion times. Since the
initialization of this cluster in August 2011, there have been around 40000 job submissions.

\paragraph{CTC IBM-SP2 Cluster}
This workload contains parallel job information for the 512-node IBM-SP2 Cluster at Cornell
Theory Center(CTC). The workload includes jobs submitted between July 1996 and May 1997 and
for each job, it provides the identifier of the user who submitted the job, in addition to the
submission and completion times.

\paragraph{LAIR62 NFS Workload}
This workload includes the NFS transactions for the first nine days of the \textit{LAIR62} workload,
a research workload from a university computer science department. The workload has information
about the submission and completion times of each NFS transaction, in addition to the source and
destination addresses and type of each transaction (response or call). 

\subsubsection{Test Procedure}
The first step of each test case is to retrieve the necessary information from each workload. A script
that goes through all the workloads parses the user identifiers and the submission and completion times
of each job in the Hornet Cluster or CTC IBM-SP2 Cluster. For the LAIR62 NFS workload, the script
parses the submission and completion times of each transaction, in addition to the source and
destination addresses of each transaction since the user identifier is not available.

Once the necessary information is parsed from the workloads, a simulation script starts assigning users
to I/O nodes using the sequential or distributed node assignment methods. It is possible to change the
total number of nodes, the number of nodes assigned to each user, the assignment method, the inactivity
threshold and the startup time in this simulation script for various evaluation scenarios.

Once the user assignment to I/O nodes is completed, the simulation script checks whether there is
any overlap between the jobs or transactions executed on an I/O node. Since we would like to know how
long each I/O node is going to stay on or off, it is important to handle the overlaps between jobs or
transactions. Assume Job1 and Job2 are assigned to I/O server $A$. If Job1 starts at time $d1$ and finishes
at $d2$, and if Job2 starts at $d3$ and finishes $d4$, these two jobs will overlap in time when $d1 < d3$
and $d3 < d2$. Thus, the total time that I/O node $A$ is kept on will be ($d2$ - $d1$) + ($d4$ - $d3$)
- $overlap time$.

Finally the simulation script calculates the output information in order to understand how well the
system is doing in terms of the energy consumption and system utilization. The simulation finds the
energy consumption and system utilization for various values of the inactivity threshold, startup time,
total number of nodes and number of nodes assigned to each user as shown in Table~\ref{tbl:params}.
Power consumption per node is 167 Watts~\cite{knightshift} for LAIR62 NFS workload and 300 Watts
~\cite{DBLP:journals/computer/BianchiniR04} for Hornet and CTC IBM-SP2 Clusters.

\begin{table}
  \begin{center}
    \begin{tabular}{|l|l|} \hline
Test Parameter & Values \\ \hline
Total number of nodes                   & 16, 32 or 64 \\ \hline
Number of nodes assigned                & 2, 4 or 8 \\
to each user                            & \\ \hline
Assignment methods                      & Sequential or Distributed \\ \hline
Low-energy mode                         & Turn Off \\ \hline
Inactivity threshold                    & Varied between 1000 and 2000000 s \\ \hline
Startup time                            & 10, 60, 120 or 240 s \\ \hline
Power consumption per node              & 167, 300 W \\ \hline
    \end{tabular}
  \end{center}
  \caption{Test Parameters}
  \label{tbl:params}
\end{table}
