In this work, we presented three different energy-aware node allocation
methods that take advantage of storage network incasting and exploit user
metadata for allocating cloud storage system resources for each user, while
adhering to load-balancing parameters on demand. We also presented a mathematical
model to estimate the outcome of the proposed methods and showed that the estimations
of this model closely match with the simulation results. Each method has been
evaluated both theoretically and through simulation-based tests using real-world
workloads. Theoretical analysis results show that our proposed methods
can provide 2-approximation solutions for minimizing energy consumption
and balancing system load (storage space usage or on-time). Simulation-based
tests show that the energy savings are as high as 60\%. The simulation-based 
tests further show that as the storage system moves from the Fixed scheme to
Dynamic Greedy scheme and then to Correlation-Based scheme, energy savings, latency
per access and coefficient of variation of the storage space and on-time increase. Therefore,
these schemes can be implemented in a cloud storage system depending on parameters
of importance (energy consumption, latency or load balancing). As an example, in
a storage system where energy consumption is the primary concern, Correlation-Based
scheme can be used. Similarly, in a storage system where load balancing and energy consumption
are equally important, Dynamic Greedy scheme can be used. We should also point
out that the latency per access values for any of the methods we proposed were
usually less than a second, which should be acceptable for most of the cloud
storage applications.

Our methods are different from related studies as we classify and place users; rather
than classifying and placing data. Additionally,
the methods we propose take load-balancing and data transfer costs into account,
which is not included in many related studies. We also have a lightweight algorithm
to predict future which is another concept that is missing in related studies, as
they mostly try to predict future reactively by monitoring the system with complex
mechanisms and possibly introducing overhead as a result of this.

The methods we presented in this work can be implemented in any type of general
storage system; including archival and parallel HPC storage systems. We evaluated
these methods with theoretical and simulation-based evaluations. A more insightful
evaluation of these methods can be conducted on a real cloud storage system. The
proposed methods were tested with up to 64 storage nodes, where 8 storage nodes at
maximum were allocated per user. Additionally, the workloads we used had at most
645 users. In a real implementation these numbers might be different; therefore,
further evaluation in a real storage system with different workloads would be also
helpful in this respect. Moreover, we assumed that the low-energy mode for a
storage node is turning it off completely. We are aware that modern disks support
various operating modes with different power requirements; therefore, it would be
interesting to evaluate our methods with disks supporting multiple operating modes.
Future research directions also include implementing the proposed methods with various
types of user-metadata other than the ones used in this work (system usage patterns,
user identifiers etc.) and to investigate the effectiveness of the proposed methods
in a storage system with mechanisms already preventing incasting.
