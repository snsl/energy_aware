The energy consumption of cloud systems has been studied extensively by the research
community. Srikantaiah et al. reduce the energy consumption of cloud systems
with workload consolidation while trying to find optimal performance
energy trade-off points~\cite{Srikantaiah:2008:EAC:1855610.1855620}.
In \cite{Kim:2009:PPC:1657120.1657121}, authors present energy aware provisioning
of virtual machines in a cloud system. Harnik et al. investigate how cloud storage
systems can operate at low-power modes while maximizing data availability and the
number of nodes to power down~\cite{Harnik:2009:LPM:1586640.1587438}. Duy et al.
propose a green scheduling algorithm to predict the future load in a cloud
system and to turn off unused nodes~\cite{5470908}. CloudScale reserves resources
based on usage in a multi-tenant cloud to reduce energy consumption~\cite{Shen:2011:CER:2038916.2038921}.
Rabbit is a distributed file system trying to save energy by turning off nodes while
making sure that at least primary data replicas are available~\cite{Amur:2010:RFP:1807128.1807164}.
Beloglazov et al. present a model to detect an overloaded host and dynamically reallocate
the virtual machines on that host for improved energy efficiency and
performance~\cite{Beloglazov:2013:MOH:2498743.2498939}

There also have been numerous studies to reduce the energy consumption of storage
and file systems in general. Most existing energy saving techniques for these systems
attempt to move less frequently used data to a subset of the nodes. Massive Array of Idle
Disks~\cite{Colarelli:2002:MAI:762761.762819} (MAID) forms two groups of storage nodes in the
system - \textit{active} and \textit{passive}. New requests are typically handled by the active
nodes, and if not, they are forwarded to the passive nodes. MAID's performance, however, is
dependent on the workload and cache characteristics.
Popular Data Concentration~\cite{Pinheiro:2004:ECT:1006209.1006220} (PDC) is another similar
technique where frequently accessed data is migrated to a group of storage nodes, called
\textit{active} nodes. Before migrating data, PDC needs to predict the future load for each storage
node. Although performing better than MAID for small workloads, PDC suffers from the overhead of
data migration and load prediction.
Wildani et al. present a technique that identifies and brings together data blocks in a workload
for better energy management, based on the likelihood of related access~\cite{Wildani:2011:EIW:1987816.1987823}.
GreenHDFS uses a hot\&cold zone approach, where frequently accessed data is located on the storage
nodes in the hot zone and unpopular data is located on the storage nodes in the cold
zone~\cite{Kaushik:2010:GTE:1924920.1924927}.
Lightning is an energy-aware cloud storage system that divides the storage nodes into hot\&cold zones
with data-classification driven data placement~\cite{Kaushik:2010:LSE:1851476.1851523}. The purpose of dividing
the storage nodes into logical hot\&cold zones is to increase the idleness in the storage system.
There have been other relevant studies that aimed directly at making better use of idle periods in a storage system.
Mountroidou et al. presents a framework that identifies when and for how long to activate a power-saving
mode to meet given performance\&power constraints~\cite{10.1109/IGCC.2011.6008570}. They also propose adaptive
workload shaping to make use of the idle periods in a workload better~\cite{Mountrouidou:2011:AWS:1958746.1958766}.
Write-offloading technique shows that enterprise workloads have idle periods as well and these periods
can be increased further by offloading writes on spun-down disks to persistent storage~\cite{Narayanan:2008:WOP:1416944.1416949}.
SRCMap is another technique where the workload is selectively consolidated on a subset of storage
nodes, proportional to the I/O workload~\cite{Verma:2010:SEP:1855511.1855531}. These data-classification driven
placement techniques work well only if one is able to predict data usage and idle period with reasonable
accuracy.

Hardware based techniques can also help with energy utilization but is
not broadly applicable.
Barroso et al. proposed that server components, particularly memory and disk
subsystems, need improvements to consume power proportional to their utilization levels~\cite{33387}.
Hibernator uses disks that can operate at different speeds to reduce energy consumption while trying to
meet performance goals~\cite{Zhu:2005:HHD:1095810.1095828}.

Architectural or file system optimizations present another opportunity
to save energy.
Ganesh et al.~\cite{GaneshWeatherspoonBalakrishnanBirman07_OptimizingPowerConsumptionInLargeScaleStorageSystems}
has shown that the Log Structured Filesystem (LFS) can be used to reduce energy consumption, since
the write requests are recorded in a log file making it possible to know on the client side which
storage node will handle the write request. This approach suffers from the overhead of cleaning
the log file.
Leverich and Kozyrakis present a technique to reduce the energy consumption of Hadoop clusters using
covering subsets to ensure data availability~\cite{Leverich:2010:EEH:1740390.1740405}. They find a trade-off
between energy savings and overall performance of the system.
Pergamum is a distributed archival storage system that saves energy by avoiding centralized controllers
~\cite{Storer:2008:PRT:1364813.1364814}.
Zhu et al. proposes power-aware storage cache management algorithms to keep the disks in low-power modes
for longer~\cite{Zhu:2004:REC:1072448.1072462}.
Diverted Access is another technique that exploits the redundancy in the storage systems to reduce
energy consumption~\cite{Pinheiro:2006:ERC:1140277.1140281}.

In more recent related studies, Chen et al. present the \textit{k-out-of-n computing} framework~\cite{6847230}
with the goal of increasing fault-tolerance and energy-efficiency during storage system access and data
processing. Eventhough, the random and greedy approaches used during the evaluations is similar
to the methods we will propose, this work is tailored for mobile devices in a dynamic network and unlike our
study, it is not concerned with load balancing. In~\cite{Collotta2015137}, the authors propose a fuzzy logic
approach that tries to improve the energy-efficiency of Bluetooth Low Energy (BLE) network used in many
Internet-of-Things environments, by predicting sleeping periods of devices in BLE network using their battery
levels and throughput-to-workload ratios. Eventhough, the scope and parameters of this work is completely
different than our approach, the authors show a method to benefit from idleness in a system using data from
system components. Finally, Sallam et al. present a proactive workload manager that tries to avoid bursty loads and
underutilization of resources that might be caused by a reactive workload manager in a virtual environment~\cite{Sallam:2014:PWM:2658292.2658555}.
They proactively predict the future state of VMs by analyzing the recently observed patterns. This approach
is similar to the future prediction method we will propose in Dynamic Greedy and Correlation-Based schemes; although,
it is tailored for virtual environments.

To the best of our knowledge, there has not been any related study trying to reduce energy consumption in 
a cloud storage system by using user metadata. The most relevant to our work is the approach by Wildani et al.
to group semantically-related data across the same set of devices to reduce the number of disk accesses resulting
in disk spin-ups. However, they group related data, while we group related users together~\cite{5668053}.

