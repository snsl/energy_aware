In this section, we present a mathematical model to estimate the outcome
of the proposed methods with varying parameters. We are particularly interested
in estimating the energy consumption, latency per access and load balance as the outcomes
of the system. The methods we proposed distribute jobs in a workload across
all of the storage nodes, eventually creating time interval series on each
storage node. The main driving parameter of the mathematical model is these series of
time intervals. We consider three different time intervals;
\textit{interjobs, interarrivals} and \textit{job lengths}. Interjob is
the interval between the completion of a job and the start of the next
job. Interarrival is the interval between the start of a job and the
start of the next job. Job length is the interval between the start and
completion of a job. We collect these time series data for each workload
and storage node and fit them to probability distribution models in order
to estimate the energy consumption, latency per access and load balance.

In the proposed mathematical model, time series can be fitted to different
probability distribution models depending on the workload and the number
of jobs assigned to a storage node. As each workload can be fitted to
different probability distribution models on each storage node, we use
$F(t)$ to represent the cumulative distribution function (CDF) at this
point for ease of presentation. By definiton of the cumulative distribution function, $F_{X}(t)$ represents
the probability that the random variable $X$, i.e. the interarrival
time between jobs, is smaller than or equal to $t$.
$F(t)$ is replaced with the actual CDF
of each workload in Section~\ref{validate_model}.
%Similarly, $F(X+1) - F(X)$ represents the probability that the random variable
%$t$ is equal to $X$.

\subsection{Energy Consumption}
We first estimate the energy consumption of a storage node.
The average job length
on a storage node without taking any latencies into account can be
calculated by
looking at the jobs on a storage node and taking their average as shown in 
Equation~\eqref{eq_math_one}, where $N_{J}$ is the number of jobs on a
storage node, $J_{avg}$ is the average job length without latencies and $J(k)$
is the length of an individual job.

\begin{equation}
J_{avg} = \frac{\sum\limits_{k=1}^{N_{J}} J(k)}{N_{J}}
\label{eq_math_one}
\end{equation}  
\hfill

In order to estimate the impact of job latencies,
we consider four different cases for the interjob
times:

\begin{itemize}
\item The interjob time
is greater than the sum of the inactivity threshold ($T_I$) and the
startup time ($T_S$).
In this case, even if the job finishing before the interjob period experiences
latency, the interjob time will still be greater than the inactivity threshold and
therefore will cause the storage node to be turned off. In this case,
the next job
assigned to this storage node will experience latency equal to the startup
time, since it will have to wait for the storage node to be turned
on. The average latency impact is the probability that the interjob time
is larger than $(T_I+T_S)$ multiplied by $(T_I+T_S)$, or
$(1-F_{T_I+T_S}(t))*(T_I+T_S))$.

\item The interjob time
is greater than the inactivity threshold, but smaller than or equal to the sum of
the inactivity threshold and startup time. In this case, if the job finishing
before the interjob period experiences latency, the interjob time might become smaller
than the inactivity threshold. This will cause the storage node not to be turned
off at all and therefore to miss the opportunity to reduce energy consumption.
The average latency impact is the probability that the interjob time
is between $T_I$ and $T_I+T_S$ multiplied by
the interjob time, or $\int_{T_I}^{T_I+T_S}tF'(t)dt$.

\item The interjob time is
smaller than or equal to the inactivity threshold, but greater than the startup time. In this
case, even if the job finishing before the interjob period experiences latency,
the storage system will stay on until the next job arrives.
The average latency impact is the probability that the interjob time
is between $T_S$ and $T_I$ multiplied by
the interjob time, or $\int_{T_S}^{T_I}tF'(t)dt$.

\item The interjob time
is smaller than or equal to the startup time. In this case, if the job
finishing before the interjob period experiences latency, it might overlap
with the next job assigned to the storage node, as the latency it experiences
(startup time) will be greater than the interjob time. If there is an overlap
between jobs, that means we are reducing energy consumption.
The average latency impact is the probability that the interjob time
is less than $T_S$ multiplied by
the interjob time, or $-\int_{0}^{T_S}tF'(t)dt$.
\end{itemize}  

Putting these impacts together, we can estimate the average
job length with latencies as 
shown in Equation~\eqref{eq_math_two}.

\begin{equation}
\label{eq_math_two}
\begin{aligned}
J_{avg}^{'} = J_{avg}\ \ &+ (1-F_{T_I+T_S}(t))*(T_I+T_S) \\
                       &+ \int_{T_I}^{T_I+T_S}tF'(t)dt \\
                       &+ \int_{T_S}^{T_I}tF'(t)dt \\
                       &- \int_{0}^{T_S}tF'(t)dt
\end{aligned}
\end{equation}
\hfill

The total time ($T$) a storage node stays on (on-time) can be estimated by using
the average job length with latencies ($J_{avg}^{'}$) and the number of jobs
on a storage node ($N_J$). Then the total time can be multipled with power
of a storage node to find the energy consumption, as shown in Equation~\eqref{eq_math_three}.

\begin{equation}
\begin{gathered}
Energy\ \ consumption\ \ of\ \ a\ \ node = Power\ \ *\ \ T,\\
where\ \ T\ \ = N_J\ \ *\ \ J_{avg}^{'}
\end{gathered}
\label{eq_math_three}
\end{equation}
\hfill

\subsection{Load Balancing}
Equation~\eqref{eq_math_three} shows the total time a storage node stays on;
where $N_J$ is the number of jobs on that storage node and $J_{avg}^{'}$
is the average job length with latencies. While balancing the on-time across
all storage nodes, we consider the coefficient of variation (CV) of on-time,
which is denoted by $CV_{T}$. Therefore, we can formulate balancing the
on-time across storage nodes with Equation~\eqref{eq_math_four}.

\begin{equation}
\label{eq_math_four}
CV_{T} = \frac{\sqrt[2]{\frac{\sum\limits_{i=1}^N (T_i - T_{mean}) ^ 2}{N}}}{T_{mean}};
\end{equation}
\hfill

where $T_{mean}$ is found as follows, with $N$ being the number of storage nodes
and $T_i$ being the total time storage node $i$ stays on;

\begin{equation}
\label{eq_math_five}
T_{mean} = \frac{\sum\limits_{i=1}^N T_i}{N};
\end{equation}
\hfill

Balancing storage space across storage nodes is represented with the same
model shown in Equation~\eqref{eq_math_four}
and~\eqref{eq_math_five}. The only difference will be using storage
space instead of on-time. We model balancing the storage space across storage nodes with
Equation~\eqref{eq_math_six}. $CV_{S}$ denotes the coefficient of
variation of storage space across all nodes.

\begin{equation}
\label{eq_math_six}
CV_{S} = \frac{\sqrt[2]{\frac{\sum\limits_{i=1}^N (S_i - S_{mean}) ^ 2}{N}}}{S_{mean}};
\end{equation}
\hfill

where $S_{mean}$ is found as follows, with $N$ being the number of storage nodes
and $S_i$ being the used storage space on node $i$;

\begin{equation}
\label{eq_math_seven}
S_{mean} = \frac{\sum\limits_{i=1}^N S_i}{N};
\end{equation}
\hfill
 
\subsection{Latency per Access}
As discussed before, any job
that is assigned to a turned-off storage node needs to wait for that node
to startup, an operation that will take time equal to $T_S$.
However, any subsequent job that is assigned to the same storage node before that
node fully starts up will also experience latency. The latency
experienced by these subsequent jobs will be smaller than $T_S$;
therefore, we need to estimate the effective latency of jobs arriving
to a storage node while that node is being started. In order to estimate
the number of jobs arriving to a storage node in a certain period of time and
to find out the latency experienced by each, we need to examine the
interarrival times of a workload. After fitting interarrivals to
probability distribution functions, we can estimate how many jobs will
arrive to a storage node when it is being started.

The first job that triggers the start of a turned-off storage node will
always experience a latency that is equal to $T_S$. Any subsequent
jobs that arrive to that storage node before it is fully turned on will experience
a smaller amount of latency. Therefore, assuming a new job arrives every
second, we can formulate the effective latency of a storage node as shown
in Equation~\eqref{eq_math_eight}.

\begin{equation}
\label{eq_math_eight}
t_{eff} = T_S + \int_{0}^{T_S} (T_S -
t) * F'(t) dt
\end{equation}
\hfill
 
In order to estimate latency per access, we need to take the average of
effective latency values on all storage nodes. This is shown in
Equation~\eqref{eq_math_nine}, where $t_{eff}(i)$
is the effective latency on storage node $i$, $N_I(i)$ is the number
of interjob periods on storage node $i$, $P_I(i)$ is the probability of an interjob period
being greater than the inactivity threshold (therefore causing the next job to
experience latency) and $N$ is the number of storage nodes in the system.

\begin{equation}
\begin{gathered}
Latency\ \ per\ \ access = \frac{\sum\limits_{i=1}^N t_{eff}(i) * N_I(i) * P_I(i)}{N},
\\
\\
where\ \ P_I\ \ =\ \ 1\ \ -\ \ F_{T_I}(t)\ \ on\ \ a\ \ storage\ \ node
\end{gathered}
\label{eq_math_nine}
\end{equation}
\hfill
