In this section we give theoretical analysis of the proposed methods in order to learn
the approximation factor of each to the best possible solution in terms of energy consumption
or load balancing. We calculate the approximation factor ($C$) of each method by comparing its
energy consumption or load balancing value with the optimal value. $C$ is the upper bound on the
worst case performance of our methods; such that, when the optimal solution is multiplied
by $C$, it is guaranteed to be equal to or greater than solutions provided by our methods.
$C$ can be formulated as shown in~\eqref{eq_th_zero}.

\begin{equation}
P_{proposed} <= C * P_{optimum}
\label{eq_th_zero}
\end{equation}
\hfill

where $C$ is the approximation factor, $P_{proposed}$ is the solution provided by one of our
methods and $P_{optimum}$ is the optimal solution.

\subsection{Load Balancing}
\subsubsection{Storage Space}
Our proposed methods try to balance storage space across storage nodes by trying to
distribute new or transferred users as evenly as possible. We assume that the optimal solution
knows the storage space usage of each user beforehand; so that, it makes the best decision in 
terms of storage space balancing when a user comes to the system for the first time or when its
storage is transferred between the storage nodes.

We assume that there are $N$ cloud storage nodes, where $M$ of them are allocated per user and
there are $K$ total users using the system. $S_{node}(i)$ represents the total eventual storage space
usage on storage node $i$, $S_{node}(i)(t)$ represents the total storage space usage
on storage node $i$ at time $t$ and $S_{user}(i)$ represents the storage space usage of user $i$. The optimal
solution will distribute users as evenly as possible, such that, the makespan of the cloud storage
system (maximum storage space usage on a storage node in the entire system) is minimized~\cite{tu_lecture}.

There are two boundaries for the optimal solution. It will either be able to allocate all $N$
storage nodes or only one storage node per user. We denote these two cases as \textit{Case1}
and \textit{Case2} and use the maximum of these two as the optimal solution for storage space balancing
$P_{optimum}$ as shown in Equation~\eqref{eq_th_three}.

\begin{itemize}
\item \textit{Case1} The optimal solution allocates all $N$ storage nodes per user. In this case,
the storage space is perfectly distributed and the makespan will be found as shown in Equation~\eqref{eq_th_one}.

\begin{equation}
P_{optimum1} = \frac{\sum\limits_{i=1}^{K} S_{user}(i)}{N}
\label{eq_th_one}
\end{equation}
\hfill

\item \textit{Case2} The optimal solution allocates only a single storage node per user. In this
case, the makespan of the storage system will be found as shown in Equation~\eqref{eq_th_two}.

\begin{equation}
P_{optimum2} = \max{S_{user}(i)},\ \ where\ \ 1 <= i <= K
\label{eq_th_two}
\end{equation}
\hfill
\end{itemize}

\begin{equation}
P_{optimum} = \max{P_{optimum1}, P_{optimum2}}
\label{eq_th_three}
\end{equation}
\hfill

When a new or transferred user comes to the cloud storage system, all but two of our proposed
methods might allocate the storage nodes with maximum storage space usage to this user.
We denote the solution of these methods as $P_{proposed\_worst}$. The two cases that will try
to achieve the best possible storage space distribution are balancing technique with storage
space balancing weight ($S_W$) set to one and Dynamic Greedy scheme that has balancing technique
as the initial technique with storage space balancing weight again set to one and energy consumption
weight ($E_W$) set to zero. Similarly, we denote the solution of these two methods as $P_{proposed\_best}$.

We first analyze the approximation factor of $P_{proposed\_best}$. In this case, two methods
mentioned in the previous paragraph allocate $M$ storage nodes with the minimum storage space usage
to a user. As it will not affect our theoretical analysis, we assume $ M = 1 $. We also assume that
storage node $f$ has the biggest storage space usage at time $T_1$ as in Equation~\eqref{eq_th_four}, after
it was allocated for user $u$ that has a storage space usage of $S_{user}(u)$.

\begin{equation}
S_{node}(f)(T_1) = \max{S_{node}(i)(T_1)},\ \ where\ \ 1 <= i <= N
\label{eq_th_four}
\end{equation}
\hfill

This means that at time $T_2$ ($T_2 < T_1$), when the decision to allocate storage node $f$ for
user $u$ was made, node $f$ had the minimum storage space usage in the system as in Equation~\eqref{eq_th_five}.

\begin{equation}
N * S_{node}(f)(T_2) <= \sum\limits_{i=1}^{N}S_{node}(i)(T_2)
\label{eq_th_five}
\end{equation}
\hfill

As the total storage space in the system at time $T_2$ will be less than the total eventual storage
space usage, we can infer Equation~\eqref{eq_th_six}.

\begin{equation}
\sum\limits_{i=1}^{N}S_{node}(i)(T_2) < \sum\limits_{i=1}^{N}S_{node}(i)
\label{eq_th_six}
\end{equation}
\hfill

Another observation we can make is that the total eventual storage space usage on the system
will be equal to the sum of individiual storage space usage of each user as shown in Equation~\eqref{eq_th_seven}. 

\begin{equation}
\sum\limits_{i=1}^{N}S_{node}(i) = \sum\limits_{i=1}^{K}S_{user}(i)
\label{eq_th_seven}
\end{equation}
\hfill

And from Equation~\eqref{eq_th_one},~\eqref{eq_th_two} and~\eqref{eq_th_three}, we can infer that;

\begin{equation}
\sum\limits_{i=1}^{K}S_{user}(i) <= N * P_{optimum}
\label{eq_th_eight}
\end{equation}
\hfill

Therefore, combining Equation~\eqref{eq_th_five},~\eqref{eq_th_six},~\eqref{eq_th_seven} and
~\eqref{eq_th_eight}, we can obtain;

\begin{equation}
S_{node}(f)(T_2) < P_{optimum}
\label{eq_th_nine}
\end{equation}
\hfill

We now know that storage node $f$ has the maximum amount of storage space used in the system at
time $T_1$ and the minimum amount of storage space used in the system at time $T_2$ (before it
was allocated for user $u$ that has a storage space usage of $S_{user}(u)$). Thus, we can infer
the equality in Equation~\eqref{eq_th_ten}.

\begin{equation}
S_{node}(f)(T_1) = S_{node}(f)(T_2) + S_{user}(u)
\label{eq_th_ten}
\end{equation}
\hfill

We can also observe the following from Equation~\eqref{eq_th_one},~\eqref{eq_th_two} and~\eqref{eq_th_three}.

\begin{equation}
S_{user}(u) <= P_{optimum}
\label{eq_th_eleven}
\end{equation}
\hfill

Therefore, combining Equation~\eqref{eq_th_nine},~\eqref{eq_th_ten} and~\eqref{eq_th_eleven}
we can see that two of the proposed methods (balancing technique with $S_W = 1$ and
Dynamic Greedy scheme that has balancing technique as the initial technique $S_W = 1$ and
$E_W = 0$) have an approximation factor of 2 in the best case.

\begin{equation}
P_{proposed\_best} = S_{node}(f)(T_1) < 2 * P_{optimum}
\label{eq_th_twelve}
\end{equation}
\hfill

We now analyze the approximation factor of other methods, $P_{proposed\_worst}$. These methods
might allocate $M$ storage nodes with the maximum storage space usage to a user in the worst case. We again
assume $ M = 1 $ as it will not change the theoretical analysis. The theoretical analysis of $P_{proposed\_worst}$
will be similar to that of $P_{proposed\_best}$. In fact, the only difference will be to Equation~\eqref{eq_th_five},
as the storage node with the maximum storage usage at time $T_2$ will definitely have storage space usage less
than or equal to the total storage space of all nodes as shown in Equation~\eqref{eq_th_thirteen}.

\begin{equation}
S_{node}(f)(T_2) <= \sum\limits_{i=1}^{N}S_{node}(i)(T_2)
\label{eq_th_thirteen}
\end{equation}
\hfill

This will change Equation~\eqref{eq_th_nine} as follows;

\begin{equation}
S_{node}(f)(T_2) < N * P_{optimum}
\label{eq_th_fourteen}
\end{equation}
\hfill

Therefore, we can see that the proposed methods have an approximation ratio of $N + 1$ in the worst case, where
$N$ is the total number of storage nodes in the system.

\begin{equation}
P_{proposed\_worst} = S_{node}(f)(T_1) < (N + 1) * P_{optimum}
\label{eq_th_fifteen}
\end{equation}
\hfill

\subsubsection{On-Time}
Theoretical analysis of our proposed methods in terms of balancing on-time across storage
nodes will be exactly the same as the theoretical analysis of balancing storage space; except
for the following;

\begin{itemize}
\item On-time information will be used instead of storage space.
\item Techniques producing $P_{proposed\_best}$ and $P_{proposed\_worst}$
will be different.
\end{itemize}

The balancing technique with on-time balancing weight ($T_W$) set to one and Dynamic Greedy scheme
that has balancing technique as the initial technique with on-time balancing weight set to one
and energy weight ($E_W$) set to zero will produce a 2-approximation $P_{proposed\_best}$ solution,
whereas other methods will produce an (N + 1)-approximation $P_{proposed\_worst}$ solution
compared to the optimal solution in terms of balancing on-time.

\subsection{Energy Consumption}
We compare the energy consumption of the proposed methods with that of the
optimal solution in order to find the approximation factor of each. We assume
again there are $N$ storage nodes in the system and $M$ of them are allocated
for each user. The theoretical model for the energy consumption is based
on \textit{interjob} times that were described earlier in Section~\ref{model}.

In the best case, the storage node that is allocated for a user by our proposed
methods will be already on. In this case, there will not be any difference
between the optimal solution and the solution provided by our provided methods.

The worst case for any of the methods we propose will be when all of the
storage nodes have been turned off due to staying inactive for longer than
the inactivity threshold; such that, a job submitted to the $M$ storage nodes
at this time will experience latency equal to $T_I + T_S$ regardless
of the method we are using. $T_I$ represents the inactivity threshold and
$T_S$ represents the startup time of a storage node.

On the other hand, the optimal solution knows when an idle period will begin and
end. Therefore, it can turn off a storage node without waiting for the inactivity
threshold to elapse. As a result, $P_{proposed}$, $P_{optimum}$ and $C$ can be
formulated as shown in Equation~\eqref{eq_th_sixteen}, \eqref{eq_th_seventeen}
and \eqref{eq_th_eighteen}.

\begin{equation}
P_{proposed} = \sum\limits_{i=1}^{M} (T_{I} + T_{S})
\label{eq_th_sixteen}
\end{equation}
\hfill

\begin{equation}
P_{optimum} = \sum\limits_{i=1}^{M} T_{S}
\label{eq_th_seventeen}
\end{equation}
\hfill

\begin{equation}
\begin{gathered}
C >= \frac{(T_{I} + T_{S})}{T_{S}} \\
C >= \frac{(k * T_{S} + T_{S})}{T_{S}} = k + 1,\ \ where\ \ k = \frac{T_{I}}{T_{S}}
\label{eq_th_eighteen}
\end{gathered}
\end{equation}
\hfill

Consequently, if any of our methods use an inactivity threshold ($T_{I}$) value
equal to the startup time ($T_{S}$), the approximation factor will be 2.

%\subsection{Latency}
%Our approaches do not keep track of the latency in the storage system and make node
%allocation decisions accordingly. Since minimizing latency is not one of the objective
%functions of our proposed methods, there will not be any approximation factor
%to an optimal solution where latency is minimized. We note that the maximum latency
%observed by a job in our proposed methods cannot be larger than the startup time ($T_S$).
%On the other hand, we have seen that a job in the optimal solution for load balancing or
%energy consumption can also observe a latency of $T_S$. Additionally, we have shown
%the actual latency per access to be much lower than $T_S$ in Section~\ref{results}.
%As discussed in Section~\ref{intro}, latency effects can be hidden too, if it is known
%a priori which nodes are going to be used at any time.
